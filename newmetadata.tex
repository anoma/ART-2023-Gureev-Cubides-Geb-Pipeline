
% Title
\newcommand{\pubtitle}{Geb Pipeline}

\newcommand{\pubauthA}{Artem Gureev}
\newcommand{\pubaffilA}{a}
% \newcommand{\orcidA}{0000-0001-5477-1503}
\newcommand{\authemailA}{artem@heliax.dev}
% \newcommand{\eqcontribA}{}

\newcommand{\pubauthB}{Jonathan Prieto-Cubides}
\newcommand{\pubaffilB}{a}
% \newcommand{\orcidB}{0000-0001-0000-0000}
\newcommand{\authemailB}{jonathan@heliax.dev}
% \newcommand{\eqcontribB}{}

% \newcommand{\pubauthC}{Last Author}
% \newcommand{\pubaffilC}{a}
% \newcommand{\orcidC}{0000-0001-5477-1503}
% \newcommand{\authemailC}{mail@someinstitute.com}

% Institutions/Affiliations
\newcommand{\pubaddrA}{%
Heliax AG }

% Corresponding author mail
\newcommand{\pubemail}{\authemailA}

\newcommand{\pubabstract}{
At Heliax, we are developing a compiler stack to facilitate the creation of
decentralized applications using high-level functional programming
languages. This stack comprises a series of compilers that begin with
Juvix and culminate in various targets. One such target is arithmetic circuits, represented via VampIR,
an intermediate language for such circuits. This report highlights the
Geb project, a component of this pipeline, and details the process of
compiling JuvixCore into VampIR through the Geb compiler. To aid
its adoption and implementation, we provide a categorical overview of the
mathematical foundations of the Geb project and insights into its
current Lisp-based implementation. The objective of this report is to guide
future implementations and improvements of the Geb project.
}

% Description of the SI file, placed as a footnote
% \newcommand{\pubSI}{Electronic Supplementary Information (ESI) available:
% one PDF file with all referenced supporting information.}

% Any keywords to be displayed under the abstract
\keywords{ 
Geb \sep%
Juvix \sep%
VampIR \sep%
compilers \sep%
category theory \sep%
semantics of PL \sep%
Lambda calculus \sep% 
Arithmetic circuits\sep}

% Supplementary space between title/abstract and text, if needed
% \newcommand{\pubVadj}{0pt}

% ! DO NOT REMOVE OR MODIFY !
\input{templates/ART/aux-preamble.tex}
% The preprint DOI to be used as an link in the paper
\pubdoi{10.5281/zenodo.8262747}
\history{(Received August 8, 2023; Published: August 22, 2023)}