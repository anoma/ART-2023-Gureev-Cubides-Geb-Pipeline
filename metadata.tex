\jnlPage{1}{00}
\jnlDoiYr{2023}

\releaseval{\today}
\updatedval{\today}

\history{(Received August 8, 2023; Published: August 21, 2023)}
\doival{10.5281/zenodo.8262747}

\title{Geb Pipeline v0.5.0}

\begin{authgrp}
\author{Artem Gureev}
\author{\;Jonathan Prieto-Cubides}
\affiliation{Heliax AG \email{artem@heliax.dev},\,\email{jonathan@heliax.dev}}
\end{authgrp}



% % Title
% \newcommand{\pubtitle}{Geb Pipeline v5.0}

% % \begin{abstract}
% % At Heliax, we are developing a compiler stack to facilitate the creation of
% % decentralized applications using high-level functional programming
% % languages. This stack comprises a series of compilers that begin with
% % \Juvix{} and culminate in arithmetic circuits, represented via \VampIR{},
% % an intermediate language for such circuits. This report highlights the
% % \Geb{} project, a component of this pipeline, and details the process of
% % compiling \JuvixCore{} into \VampIR{} through the \Geb{} compiler. To aid
% % its adoption and implementation, we provide a categorical overview of the
% % mathematical foundations of the \Geb{} project and insights into its
% % current Lisp-based implementation. The objective of this report is to guide
% % future implementations and improvements of the \Geb{} project.
% % \end{abstract}


% % \begin{keywords}

% % \end{keywords}
% % \maketitle


% % Authors, with name (\pubauthX), affiliation (\pubaffilX), ORCID (\orcidX)
% % and email (\authemailX) Can add authors with X = A-K Add an equal
% % contribution flag (\eqcontribX) to mark authors as equally contributed
% \newcommand{\pubauthA}{Your Namehere}
% \newcommand{\pubaffilA}{a,b}
% \newcommand{\orcidA}{0000-0001-5477-1503}
% \newcommand{\authemailA}{youremail@someinstitute.com}
% % \newcommand{\eqcontribA}{}

% \newcommand{\pubauthB}{Second Equal}
% \newcommand{\pubaffilB}{b}
% \newcommand{\orcidB}{0000-0001-0000-0000}
% \newcommand{\authemailB}{other@mail.com}
% % \newcommand{\eqcontribB}{}

% \newcommand{\pubauthC}{Last Author}
% \newcommand{\pubaffilC}{a}
% \newcommand{\orcidC}{0000-0001-5477-1503}
% \newcommand{\authemailC}{mail@someinstitute.com}

% % Institutions/Affiliations
% \newcommand{\pubaddrA}{%
% Institution A, with some address. } \newcommand{\pubaddrB}{%
% Institution B, with its other address. }

% % Corresponding author mail
% \newcommand{\pubemail}{\authemailC}

% % The publication abstract No references/commands to be placed in here
% \newcommand{\pubabstract}{
% At Heliax, we are developing a compiler stack to facilitate the creation of
% decentralized applications using high-level functional programming
% languages. This stack comprises a series of compilers that begin with
% \Juvix{} and culminate in arithmetic circuits, represented via \VampIR{},
% an intermediate language for such circuits. This report highlights the
% \Geb{} project, a component of this pipeline, and details the process of
% compiling \JuvixCore{} into \VampIR{} through the \Geb{} compiler. To aid
% its adoption and implementation, we provide a categorical overview of the
% mathematical foundations of the \Geb{} project and insights into its
% current Lisp-based implementation. The objective of this report is to guide
% future implementations and improvements of the \Geb{} project.
% }

% % The preprint DOI to be used as an link in the paper
% \pubdoi{10.5281/zenodo.8262747}

% % Description of the SI file, placed as a footnote
% % \newcommand{\pubSI}{Electronic Supplementary Information (ESI) available:
% % one PDF file with all referenced supporting information.}

% % Any keywords to be displayed under the abstract
% \keywords{ 
% Geb \sep%
% Juvix \sep%
% VampIR \sep%
% compilers \sep%
% category theory \sep%
% semantics of PL \sep%
% Lambda calculus \sep% 
% Arithmetic circuits}

% % Supplementary space between title/abstract and text, if needed
% % \newcommand{\pubVadj}{0pt}

% \input{templates/ART/aux-preamble.tex} % ! DO NOT REMOVE